\section{Klänge}\label{kluxe4nge}

Erforderliche Kompetenzen: Gewonnene Kompetenzen: KLÄNGE

\begin{figure}[ht]
    \centering 
    \includeimage{img/20.png}
    \caption[\Sectionname]{\Sectionname}
\end{figure}

\subsection{Beschreibung}\label{beschreibung}

Ziel dieses Projekts ist die Einführung in die Kategorie ``Klänge''.
Ziel ist es, eine Bühne aus Figur und einigen Musikinstrumenten
aufzubauen. Passend zu einer Hintergrundmusik kann die Figur durch
Tastendrücke, Klicks auf den Bildschirm etc. dazu veranlasst, die
verschiedenen Musikinstrumente zu spielen. Das Projekt kann sehr gut
ausgebaut werden und eignet sich vor allem für kreativ veranlagte
Schülerinnen und Schüler, da das Maß an Gestaltungsmöglichkeiten sehr
groß ist.

Unbedingt erforderlich zur Durchführung des Projekts ist es, dass PCs
mit Lautsprechern oder Kopfhörern vorhanden sind. Der Projektumfang
beträgt etwa eine Doppelstunde an der Grundschule.

\subsection{Durchführung}\label{durchfuxfchrung}

\subsubsection{Phase 1: Planung}\label{phase-1-planung}

Zum Beginn sollte das fertige Projekt gezeigt und vorgestellt werden.
Unter Umständen kann es Sinn machen, auch einzelne Schülerinnen und
Schüler die fertige Lösung am Lehrerrechner ausprobieren zu lassen.
Anschließend sollten die wichtigsten Befehle erwähnt werden. Dazu
gehören

\begin{itemize}
\item
  \textbf{Wie gelange ich zur Auswahl der Klänge?} Die geschieht durch
  die Auswahl einer Figur und Wahl der Registerkarte ``Klänge''.
  Anschließend können zur aktuellen Figur neue Klänge hinzugefügt
  werden.
\item
  \textbf{Wie spiele ich im Programm einen Klang ab?} Das Abspielen
  erfolgt mit dem Block ``spiele Klang (ganz)'' in der Kategorie
  ``Klang'' der vorher ausgewählten Figur.
\end{itemize}

Sofern es zuvor noch nicht angesprochen wurde, sollten die Schülerinnen
und Schüler zuvor zusätzlich noch damit vertraut gemacht werden, wie auf
Tastendrücke oder Mausaktionen hin mit Blöcken der Kategorie Ereignisse
auf Nutzereingaben reagiert werden kann.

\subsubsection{Phase 2: Vorbereitung}\label{phase-2-vorbereitung}

. Die Schülerinnen und Schüler sollten nun ausreichend Zeit zur
Verfügung gestellt bekommen, um sich einen Hintergrund sowie Figuren und
Instrumente auszusuchen bzw. selbst zu malen. Diese Phase darf ruhig
mehr Zeit in Anspruch nehmen. Es ist darauf zu achten, dass viele
individuelle Fragen zum Zeichnen, Einbinden von Figuren und vor allem
auch dem Einbinden von Klängen aufkommen werden.

Zum Ende dieser Phase sollte jeder Teilnehmer über ein Bühnenbild mit
einer oder mehreren Figuren verfügen. Darüber hinaus sollten in der
``Hauptfigur'' ein langer Ton für das Abspielen im Hintergrund und
einige kurze Töne (z. B. Schlagzeug) ausgewählt sein.

\subsubsection{Phase 3: Programmierung}\label{phase-3-programmierung}

Die Programmierung kann bei diesem Projekt beliebig komplex werden. Die
Reihenfolge der verschiedenen Funktionen kann dabei sehr stark von der
Lernmotivation und den Interessen der Schülerinnen und Schülern
abhängen. Generell empfiehlt es sich, bei Anmerkungen folgenden Ablauf
im Hinterkopf zu haben:

\begin{itemize}
\item
  Abspielen eines Klanges im Hintergrund
\item
  Abspielen weiterer Klänge basierend auf Interaktion (Klicken auf
  Figur, Drücken von Tasten, Berühren anderer Figuren, \ldots{})
\item
  Einbau von Animationen, z. B. Tanzbewegungen
\end{itemize}

Es sollte bei diesem Projekt stark auf die individuellen Ideen der
Schüler geachtet werden.

\subsubsection{Phase 4: Testen und
vorstellen}\label{phase-4-testen-und-vorstellen}

Am Ende sollten einige Minuten für die Vorstellung des Projekts
eingeplant werden. Nach Möglichkeit sollte jeder sein Projekt kurz
selbst präsentieren (und steuern) dürfen, sofern der Lehrerrechner
entsprechend mit Lautsprechern ausgestattet ist.

\subsection{Erweiterungen}\label{erweiterungen}

Für das Projekt ist eine ganze Reihe von Erweiterungen denkbar. So kann
die Bühne etwa mit Vorhängen versehen werden, die sich zu Beginn öffnen
und am Ende der Vorstellung wieder geschlossen werden. Dies kann
beispielsweise durch zwei ``Figuren'' geschehen, die sich auf Befehl vor
den Hintergrund und die anderen Figuren schieben.
