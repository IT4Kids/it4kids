\section{Maus zum Käse}\label{maus-zum-kuxe4se}

\textbf{Erforderliche Kompetenzen:}\\
keine

\textbf{Gewonnene Kompetenzen:}\\
BEWEGUNGEN

\begin{figure}[ht]
    \centering 
    \includeimage{img/29.png}
    \caption[\Sectionname]{\Sectionname}
\end{figure}

\subsection{Beschreibung}\label{beschreibung}

Dieses Projekt kann relativ gut skaliert werden, von einem recht
einfachen bis zu einem recht umfangreichen Schwierigkeitsgrad. Primäres
Ziel ist die Einführung von \emph{Bewegungen} und erste Erfahrung mit
den Funktionen aus der Kategorie \emph{Fühlen}. Optional kann auch mit
den grundlegenden Funktionen aus der Kategorie \emph{Steuerung}
gearbeitet werden.

Das Projekt enstand als Anlehnung an die manuell durchgeführten
Labyrinth Versuche. Als Spielfiguren kommen eine Maus und ein Stück Käse
zum Einsatz. Die Maus befindet sich auf einer Insel, die über einen
Holzsteg mit einer anderen Insel verbunden ist. Die Schülerinnen und
Schüler sollen nun versuchen, mithilfe von hintereinander ausgeführten
Bewegungen die Maus bis zum Käse zu bringen. Im ersten Anlauf können die
einzelnen Befehle dabei manuell angeklickt werden, wenn alles
funktioniert, wird alles zu einem Programm zusammengefasst.

\subsection{Erweiterung}\label{erweiterung}

Schnelle Schüler können versuchen, die Maus auf Tastendruck reagieren zu
lassen und sie somit zu steuern. Zudem können mehrere ``Level'' gespielt
werden, indem weitere Hintergrundbilder angelegt werden.
