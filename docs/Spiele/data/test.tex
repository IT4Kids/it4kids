\section{Finde das Tier}

Erforderliche Kompetenzen: Steuerung(Bedingungen, Wiederholungen);
Bewegung(Gehe zu); Variablen; Klänge; Nachrichten

Gewonnene Kompetenzen: Ergeignis: Wenn ich angeklickt werde;
Zufallszahlen

\subsection{Beschreibung}

Finde das Tier ist ein Reaktionsspiel, bei dem der Spieler innerhalb
einer bestimmten Zeit eine bestimmte, durch Zufall ausgewählte Figur
anklicken muss, um einen Punkt zu erlangen und einen Punkt verliert,
wenn er die falsche Figur anklickt. Zusätzlich bewegen sich die Figuren
zufällig im Spielfeld, um das Spiel zu erschweren.

\subsection{Durchführung}\label{durchfuxfchrung}

\subsubsection{Phase 1\&2: Planung und Vorbereitung}

Zunächst soll das Projekt den AG-Teilnehmern vorgestellt und erläutert
werden. Anschließend werden unbekannte Blöcke wie das Ereignis ``Wenn
ich angeklickt werde'' und die Zufallszahlen erklärt. Dies kann mithilfe
von kleinen Beispielprogrammen geschehen. Insbesondere soll gezeigt
werden, wie sich die Figuren im Raum zufällig bewegen können mithilfe
von ``gehe zu:'' und Zufallszahlen.

\subsubsection{Phase 3: Programmierung}

Den Schiedsrichter programmieren: Zuallererst soll der Schiedsrichter
(in der Vorlage die Katze) also die Figur, die das zu fangende Tier
bestimmt, programmiert werden. Dazu soll zunächst die Variable
``Gesuchtes Tier'' angelegt werden.

Das Skript des Schiedsrichters soll nun aus einer ``Endlosschleife''
(wiederhole fortlaufend) bestehen, in der in jedem Durchgang die
Variable ``Gesuchtes Tier'' auf einen zufälligen Wert zwischen eins und
drei gesetzt wird.

Anschließend soll nun der Schiedsrichter dem Spieler noch mitteilen
welches Tier gesucht wird. Dies kann mit einer ``falls \ldots{} dann''
Bedingung realisiert werden. (Wichtig: Hier ist der Block ``sage xxx für
1 Sekunde'' nötig, damit die Schleife nicht zu schnell läuft.)

Schiedsrichter mit Spielfiguren verbinden: Nun sollen drei Spielfiguren
erstellt werden, die die zu fangenden Tiere darstellen. Bei jedem
Durchlauf der Schleife des Schiedsrichters, sollen die Tieren eine
zufällige Position im Raum einnehmen. Dazu soll der Schiedsrichter eine
Nachricht (z.Bsp. ``Position'') an alle Tiere aussenden, bei deren
Empfang eine zufällige x-und y-Position eingenommen wird. (Am besten
geschieht der Positionswechsel vor dem Setzen der Variable ``Gesuchtes
Tier''.)

Den Punktestand programmieren: Als nächstes muss die Variable ``Punkte''
angelegt werden. Sie soll zu beginn jedes Spiels auf null gesetzt
werden. Die Änderung des Punktestandes erfolgt im Skript der
Spielfiguren, nicht des Schiedsrichters. Hier wird nun das neue Ereignis
``Wenn ich angeklickt werde'' verwendet, um den Klick auf dieses Tier zu
registrieren. Schließlich soll überprüft werden, ob das richtige Tier
angeklickt worden ist. Ist dies der Fall so soll der Punktestand um eins
erhöht werden, wenn nicht, soll er um eins verringert werden.

Das Spielende programmieren: Zu guter Letzt soll nun einprogrammiert
werden, dass das Spiel endet sobald eine bestimmte Punktzahl erreicht
worden ist. Dies geschieht innerhalb der ``wiederhole
fortlaufend''-Anweisung des Schiedsrichters, wo die Variable ``Punkte''
abgefragt wird. Falls das Spiel beendet werden soll, teilt der
Schiedsrichter der Spieler mit, dass er gewonnen hat und stoppt alle
Skripte.

\subsubsection{Phase 4: Testen und vorstellen}

In der letzten Phase, die auch zweimal durchgeführt werden kann, können
alle (bei großen Gruppen: ausgewählte) Schülerinnen und Schüler ihr
Projekt vorstellen. Hier bietet es sich an, wenn eine dritte Person das
Programm bedienen muss und nicht die Programmiererin oder der
Programmierer. Der restliche Kurs schaut bei der Vorstellung zu und gibt
hinterher eine Rückmeldung. Für jedes Projekt kann so analysiert werden,
was gut und was weniger gut funktioniert und noch verbessert werden
muss. Wichtig vor allem bei jüngeren Klassenstufen: Die guten Teile des
Programmes besonders hervorheben! Es kann anschließend eine weitere
Programmierphase angestoßen werden, bei der jedem die Chance gegeben
wird, die vorhandenen Mängel noch zu beheben.

\subsection{Erweiterungen}
