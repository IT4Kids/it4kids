\section{Zauberwürfel}\label{zauberwuxfcrfel}

Materialien: Ein Zauberwürfel(``Rubicks Cube'') Schwierigkeit: Eher für
ältere Schüler

Der Zauberwürfel sollte bis auf einige verbleibende Drehungen gelöst
sein, auf jeden Fall muss die Lösung relativ offensichtlich sein. Einem
Schüler werden die Augen verbunden, er erhält anschließend den Würfel,
den er vor sich auf den Tisch legt.

Anschließend haben die Kinder die Möglichkeit, nacheinander dem
Probanten Anweisungen zu geben. Dazu kann das vorliegende Spektrum an
``Befehlen'' genutzt werden:

\begin{itemize}
\item
  Drehe nach oben: Der Schüler dreht den Würfel so, dass die von ihm
  weg gerichtete Seite nun in Richtung Decke zeigt
\item
  Drehe nach rechts: Der Schüler dreht den Würfel so, dass die von ihm
  nach links ausgerichtete Seite nun in Richtung Decke zeigt
\item
  Verändere: Der Schüler dreht die obere ``Scheibe'' des Würfels im
  Uhrzeigersinn
\end{itemize}

Die Befehle sind bewusst auf ein Minimum reduziert, gerade so, dass das
Problem lösbar ist. Während dem Schüler die Befehle mitgeteilt werden,
schreibt die Lehrperson oder ein weiterer Schüler die Befehle an die
Tafel. Nach einiger Zeit sollte das Problem gelöst sein, wenn vielleicht
auch nicht optimal. Dann wird der Würfel an einen weiteren Schüler
gegeben (wieder entsprechend präpariert), der die Augenbinde erhält. Ein
weiterer Schüler liest das Programm vor, das anschließend umgesetzt
wird.

Optional ist es möglich, die Schüler mit der Frage zu konfrontieren, wie
der erste Zustand wiederhergestellt werden kann und wie sich das Problem
noch effizienter lösen lässt.

%// TODO lege als Tabelle an!
