\section{Farben}\label{farben}

\textbf{Erforderliche Kompetenzen:} keine

\textbf{Erworbene Kompetenzen:} Fühlen, Bedingungen

\begin{figure}[ht]
    \centering 
    \includeimage{img/33.png}
    \caption[\Sectionname]{\Sectionname}
\end{figure}

\subsection{Beschreibung}\label{beschreibung}

Bei diesem Projekt geht es darum, die verschiedenen Funktionen aus der
Klasse \emph{Fühlen} zu testen. Grundsätzlich bekommen die Schülerinnen
und Schüler einen zwei- oder mehrfarbigen Hintergrund. Auf Basis der
Hintergrundfarbe soll sich das Aussehen der Figur verändern. Dabei kann
die Figur zwischen den einzelnen Bereichen verschoben werden.

In einer Endlosschleife soll immer die aktuelle Hintergrundfarbe
abgefragt werden. Dies geschieht über die entsprechende Funktion in der
Befehlsgruppe ``Fühlen''. Über eine Falls-Bedingung kann abgefragt
werden, ob die Hintergrundfarbe der gewählten entspricht. Ist dies der
Fall, so soll eine bestimmte Aktion ausgeführt werden. Obligatorisch
wäre eine Ausgabe in einer Sprechblase, hier kann den Schülern aber
grundsätzlich Freiraum für eigene Experimente gelassen werden.

\subsection{Vorbereitung}\label{vorbereitung}

Sehr wichtig für die Durchführung dieses Projekts ist die Einführung von
Kontrollstrukturen. Das Programm später wird auf die ``Falls \ldots{}
Dann'' Abfrage zurückgreifen, diese sollte also in der Vorbereitung
vorab angesprochen werden.

Der entsprechende Block findet sich in der Kategorie ``Steuerung''. Als
Argument muss in den Block noch ein weiterer Befehl eingesetzt werden.
Im Falle dieses Projekts handelt es sich dabei um einen Block aus der
Kategorie ``Fühlen''. Hier stehen zahlreiche Funktionen bereit, so wie
die Kollisionserkennung mit anderen Objekten, dem Mauszeiger oder dem
Rand auch das Abfragen der Hintergrundfarbe, was wir uns in diesem
Projekt zunutze machen wollen.

In der Vorbereitung sollte den Schülerinnen und Schülern nun vermittelt
werden, dass mithilfe dieser Kontrollstrukturen eine Abfrage der
Hintergrundfarbe mit anschließendem Ausführen einer Aktion möglich ist.

\subsection{Programmierung}\label{programmierung}

Die erste Aufgabe ist es, die Hintergrundfarbe als Text auf dem
Bildschirm auszugeben. Dazu kann der entsprechende Befehl aus der Klasse
``Aussehen'' verwendet werden.

In der Vorlage steht ein Hintergrund mit verschiedenen Farben bereit.

Die Programmabfolge soll nun so aussehen, dass die Figur mit der Maus
auf eine Farbe gezogen wird. Sobald die Figur angeklickt wird, soll die
aktuelle Farbe ausgegeben werden.

Zur Lösung ist die Verwendung einer Kontrollstruktur pro Farbe notwenig,
gekoppelt an die jeweilige Ausgabe der Farbe.

\subsection{Erweiterungen}\label{erweiterungen}

Für dieses Projekt sind einige Änderungen denkbar: Schülerinnen und
Schüler zeichnen den Hintergrund selbst, statt die Vorlage zu
verwenden.\\
Nach Fühlen einer anderen Farbe ist eine Kostümänderung anstelle einer
Ausgabe in einer Sprechblase denkbar.
