\section{Bananen hergezaubert}\label{bananen-hergezaubert}

\textbf{Erforderliche Kompetenzen:}\\
Bewegung

\textbf{Gewonnene Kompetenzen:}\\
Klone, Senden und Empfangen, Zufall

\begin{figure}[ht]
    \centering 
    \includeimage{img/23.png}
    \caption[\Sectionname]{\Sectionname}
\end{figure}

\subsection{Beschreibung}\label{beschreibung}

Der Zauberer oben links wird mehrmals angeklickt, woraufhin er aus einer
Banane sehr, sehr viele herzaubert. Dieses Projekt bietet eine
Möglichkeit Zufall und Botschaften (in diesem Fall als ``Zauber'')
anschaulich zu machen

\subsection{Durchführung}\label{durchfuxfchrung}

\subsubsection{Phase 1: Planung}\label{phase-1-planung}

In der Planungsphase werden mit den Schülern die folgenden Dinge
besprochen:

\begin{itemize}
\item
  Wie klont man eine Figur? Wie steuert man den Klon? Was bedeuten die
  Klonbefehle unter ``Steuerung''?
\item
  Wie bewegt sich der Klon? Wie wählt er sein Ziel aus?
\item
  Was ist der Trigger für das Klonen? Welche Rolle spielt dabei der
  Zauberer?
\end{itemize}

\subsubsection{Phase 2: Vorbereitungen}\label{phase-2-vorbereitungen}

In der Vorbereitungsphase suchen sich die Schüler die Figuren aus:
Zauberer, Zauberteppich, Banane und Affe. Selbstverständlich können auch
diverse andere Kombinationen aus den verfügbaren Figuren der
Scratch-Spritebibliothek verwendet oder Schüler können sich selbst etwas
malen.

\subsubsection{Phase 3: Programmierung}\label{phase-3-programmierung}

\begin{enumerate}
\item
  Die Banane klont sich selbst
\item
  Der Klon gleitet an eine zufällig gewählte Spielfeldposition
\item
  Die Banane klont sich selbst genau dann, wenn der Zauberer angeklickt
  wird
\item
  Optional können verschiedene Geräusche und Farbeffekte an den
  passenden Stellen eingebaut werden
\end{enumerate}

\subsubsection{Detaillierte
Programmbeschreibung}\label{detaillierte-programmbeschreibung}

Die Musterlösung funktioniert folgendermaßen: Bananen erzeugt einen Klon
von sich selbst, wenn sie die Nachricht ``Teilt euch!'' erhält. Im
``When I start as a clone''-Block gleitet Bananen in zufällig bestimmt 1
bis 10 Sekunden zu einer ebenfalls zufällig gewählten Spielfeldposition.
Zauberer versendet die Nachricht ``Teilt euch!'' in dem Moment, das sie
angeklickt wird.

\subsubsection{Phase 4: Testen und
vorstellen}\label{phase-4-testen-und-vorstellen}

In der Vorstellungsphase kann ein beliebiger Schüler sein Spiel
vorstellen, falls er es möchte. Dabei kann man explizit auf die
einzelnen Teilprobleme hinweisen, welche der Schüler gelöst hat. Aber
man kann die anderen auch fragen, was er verbessern könnte, wenn er an
diesem Projekt nächste Stunde noch arbeiten will.

\subsection{Erweiterungen}\label{erweiterungen}

Es wäre denkbar, dass der Affe mit den Pfeiltasten bewegt werden kann
und dabei die berührten Bananen gegessen werden.
