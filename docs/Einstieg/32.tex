\section{Buchstaben}\label{buchstaben}

\textbf{Erforderliche Kompetenzen:} Scratch-Umgebung kennen

\textbf{Gewonnene Kompetenzen:}\\
Figuren auswählen/bearbeiten, Ereignisse, Bewegung (gehe zu, drehe
dich), Steuerung \{Wiederholungen (wiederhole)\}

\begin{figure}[ht]
    \centering 
    \includeimage{img/32.png}
    \caption[\Sectionname]{\Sectionname}
\end{figure}

\subsection{Beschreibung}\label{beschreibung}

Es soll ein Programm erstellet werden, in welchen beliebige Figuren
(beisielsweise Buchstaben) wackelnd, drehend, größer oder kleiner
werdend, \ldots{} zu einer sinnvollen Endposition (einem Wort) gleiten.

\subsection{Durchführung}\label{durchfuxfchrung}

In der Planungsphase werden mit den Schülern die folgenden Punkte
besprochen:

\subsubsection{Phase 1: Planung}\label{phase-1-planung}

\begin{itemize}
\tightlist
\item
  Welche Möglichkeiten stehen mit bei den Figuren zur Verfügung? Wie
  wähle ich diese aus?
\item
  Wie Programmiere ich eine spezielle Figur und nicht versehentlich eine
  andere?
\item
  Was sind Koordinaten und Winkel? Wie kann ich diese ablesen?
\end{itemize}

\subsubsection{Phase 2: Vorbereitungen}\label{phase-2-vorbereitungen}

In der Vorbereitungsphase suchen sich die Schüler die Figuren aus. Diese
können sie auch noch beliebig bearbeiten und ihrer Kreativität freien
Lauf lassen indem sie z.B. auch weitere Figuren selber malen.

\subsubsection{Phase 3: Programmierung}\label{phase-3-programmierung}

Die nötigen Schritte sind:

\begin{itemize}
\tightlist
\item
  Beim Start Figuren zum Chaospunkt Position springen lassen.
\item
  Figuren drehend, \ldots{} zum geordneten Position bewegen.
\end{itemize}

\subsubsection{Phase 4: Testen und
vorstellen}\label{phase-4-testen-und-vorstellen}

In der Vorstellungsphase können die Schüler ihre eigenständig
Programmierten Spiele vorstellen, um den anderen Ihre Kreativität zu
zeigen und den anderen dadurch weitere Anregungen für die nächste Stunde
zu geben.

\subsection{Erweiterungen}\label{erweiterungen}

Als Erweiterung zu Buchstaben besteht die Möglichkeit den Schülern das
Projekt mit Musik hinterlegen zu lassen, oder die Figuren mit kleinen
Änderungen mehrmals einfügen, um diese dann geschickt wechselnd eine
Bewegung Simulieren zu lassen oder einfach nur die Farbe ändernd.
