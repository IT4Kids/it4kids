\section{Zitterlabyrinth}\label{zitterlabyrinth}

Erforderliche Kompetenzen: Fühlen, Schleifen\\
Gewonnene Kompetenzen: Variablen (Zeit Stoppen), Bewegung (Mauszeiger
verfolgen)

\begin{figure}[ht]
    \centering 
    \includeimage{img/24.png}
    \caption[\Sectionname]{\Sectionname}
\end{figure}

\subsection{Beschreibung}\label{beschreibung}

Bei diesem Konzentrationsspiel soll mit der Maus eine Figur durch ein
enges Labyrinth bewegt werden, dessen Wände nicht berührt werden dürfen.
Zusätzlich soll die benötigte Zeit zum durchqueren des Labyrinths
gemessen werden.

\subsection{Durchführung}\label{durchfuxfchrung}

\subsubsection{Phase 1: Planung}\label{phase-1-planung}

Während der Planungsphase soll mit den Schülern folgende Schritte
vollzogen werden:

\begin{itemize}
\item
  Erläuterung des Blocks ``Gehe zu Mauszeiger''
\item
  Ideensammlung zu den Funktionen des Programms: Bewegung der Figur,
  Erkennung der Wandberührung
\item
  Überlegung anstellen, wie man den Mauszeiger von Startfähnchen zum
  Beginn des Labyrinths bekommen kann, ohne bereits die Wände zu
  berühren.
\end{itemize}

\subsubsection{Phase 2: Vorbereitungen}\label{phase-2-vorbereitungen}

Zunächst sollte der Hintergrund, also das Labyrinth mit farblich
gekennzeichneten Start- und Endpunkt gezeichnet werden. Anschließend
soll die Figur gezeichnet werden. Dabei ist darauf zu achten, dass das
Kostüm sich zentriert im Bild befindet, damit sie sich nicht verschoben
zum Mauszeiger bewegt.

\subsubsection{Phase 3: Programmierung}\label{phase-3-programmierung}

\begin{enumerate}
\item
  Bewegung der Figur: Die Bewegung der Figur kann mit einer simplen
  Wiederholungsschleife, die den Block ``Gehe zu Mauszeiger'' enthält,
  programmiert werden.
\item
  Kontrolle der Wandberührung: Bewegung der Figur und Wandberührung
  können als parallele Skripte programmiert werden.

  Die Kontrolle der Wandberührung darf jedoch erst einsetzen, sobald die
  Figur den Startpunkt berührt hat. Möglich ist dies mit dem ``Warte
  bis''-Block im Bereich Steuerung.
\item
  Zeitmessung: Die Zeitmessung soll ebenfalls mit der Berührung des
  Startpunkts beginnen und mit dem Berühren des Endpunktes enden. Dazu
  wird eine Variable ``Zeit'' angelegt, die den Startwert Null besitzt
  und bis zur Berührung des Endpunkts die verstrichene Zeit zählt, indem
  jede 0.1 Sekunden die Variable um 0.1 erhöht wird.
\end{enumerate}

\subsubsection{Detaillierte
Programmbeschreibung}\label{detaillierte-programmbeschreibung}

Das Programm besteht aus einem Bühnenbild und einer Figur. Das
Bühnenbild enthält das Labyrinth, durch das die Figur bewegt werden
soll. Zudem sind der Start- und der Endpunkt des Labyrinths mit
Farbpunkten versehen, damit die Zeitmessung installiert werden kann und
damit die Figur vom „grünen Fähnchen`` zum Startpunkt bewegt werden
kann, ohne mit den Wänden des Labyrinths zu kollidieren.

Zunächst muss die Bewegung der Figur programmiert werden. Da die Figur
immer dem Cursor folgen soll, wird im Skript der Figur in einer
Wiederholeschleife der Block ``gehe zu Mauszeiger'' ausgeführt.

Anschließend muss also von der Figur das Erreichen des Startpunktes (in
der Musterlösung rot) registriert werden. Dazu wird der ``warte bis wird
Farbe rot berührt''-Block verwendet. Anschließend wird mit dem ``warte
bis wird Farbe schwarz berührt''-Block die Kollision mit der
Labyrinthwand detektiert.

Die Zeitmessung wird parallel zur Wanddetektion im Skript der Figur
implementiert, und zwar mit einer Variable namens Zeit, die bei
Erreichen des Startpunktes auf Null gesetzt wird und anschließend in
einer Schleife mit der Abbruchbedingung ``wird Farbe blau berührt''
hochgezählt (blau ist Farbe des Endpunktes).

Außerdem wird die Größe der Figur bei Programmbeginn auf 100 gesetzt, da
sich bei Kollision mit der Wand die Figur aufbläht, um die Wandberührung
zu verdeutlichen.

\subsubsection{Phase 4: Testen und
vorstellen}\label{phase-4-testen-und-vorstellen}

In der letzten Phase, die auch zweimal durchgeführt werden kann, müssen
alle (bei großen Gruppen: ausgewählte) Schülerinnen und Schüler ihr
Projekt vorstellen. Hier bietet es sich an, wenn eine dritte Person das
Programm bedienen muss und nicht die Programmiererin oder der
Programmierer. Der restliche Kurs schaut bei der Vorstellung zu und gibt
hinterher eine Rückmeldung. Für jedes Projekt kann so analysiert werden,
was gut und was weniger gut funktioniert und noch verbessert werden
muss. Wichtig vor allem bei jüngeren Klassenstufen: Die guten Teile des
Programmes besonders hervorheben!

\subsection{Erweiterungen}\label{erweiterungen}

\begin{itemize}
\item
  \emph{Mehrere Levels:} eine besonders schnelle Schülerin könnte ein
  oder zwei zusätzliche Hintergrundbilder (also Labyrinthe) zeichnen und
  das Spiel auf mehrere Levels skalieren: Man beginnt mit Level 1 und
  wenn man gewinnt, wechselt das Bühnenbild und man kommt zu Level 2
  etc. Hier könnte man den Befehl ``Wechsle zu Bühnenbild ()''
  verwenden.
\item
  \emph{\ldots{}}
\end{itemize}
