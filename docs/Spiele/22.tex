\section{Pong}\label{pong}

Erforderliche Kompetenzen: BEWEGUNGEN, FÜHLEN, BEDINGUNGEN, VARIABLEN\\
Erworbene Kompetenzen:

\begin{figure}[ht]
    \centering 
    \includeimage{img/22.png}
    \caption[\Sectionname]{\Sectionname}
\end{figure}

\subsection{Beschreibung}\label{beschreibung}

Dieses Projekt gehört zu den eher fortgeschrittenen Projekten. Wie auch
beim Projekt \emph{Pacman} geht es um ein komplettes Spiel.

\subsection{Durchführung}\label{durchfuxfchrung}

\subsubsection{Phase 1: Planung}\label{phase-1-planung}

Während der Planungsphase sollten mit den Schülerinnen und Schülern die
folgenden Schritte vollzuogen werden:

\begin{itemize}
\item
  Überlegungen und Ideensammlung für Hintergrund, Figuren, etc.
\item
  Überlegungen zur Programmierung und der nötigen Aufgaben:

  \begin{itemize}
  \item
    Bewegungen der Schläger (hoch und herunter)
  \item
    Bewegung des Balls (bei jüngeren Schülern kann diese später
    vorgegeben werden, einige Überlegungen sollten in der Planungsphase
    aber dennoch angestellt werden)
  \item
    Zählen der Punktzahl
  \item
    Verhalten des Balls an Rändern und am Schläger
  \end{itemize}
\item
  Sammeln aller Ideen und Erstellung der TODO Liste
\end{itemize}

Anschließend kann mit der Umsetzung des Projekts begonnen werden.

\subsubsection{Phase 2: Vorbereitungen}\label{phase-2-vorbereitungen}

Zunächst sollten die Figuren und der Hintergrund gezeichnet werden. Für
die Figuren bieten sich einfache Striche an, die auf und ab bewegt
werden können, die Gestaltung des Hintergrunds ist den Schülerinnen und
Schülern überlassen.

\subsubsection{Phase 3: Programmierung}\label{phase-3-programmierung}

Im Anschluss an die vorbereitetenden Maßnahmen kann mit der
Programmierung des Spiels begonnen werden. Als Vorschlag sei die
folgende Reihenfolge genannt, in der Schritt für Schritt leicht
validierbare Teile des Spiels entstehen:

\begin{enumerate}
\item
  \emph{Steuerung der Figuren:} Beide Spielfiguren müssen durch
  entsprechende Tasten auf der Tastatur nach oben bzw. unten bewegt
  werden können. Da das Spiel mit zwei Spielern gespielt werden soll,
  bieten sich die Tasten Q und A für Spieler 1 und Pfeil hoch/Pfeil
  herunter für Spieler 2 an.
\item
  \emph{Bewegung des Balls:} Die Bewegung des Balls erfodert deutlich
  mehr Aufwand. Je nach Kenntnisstand der Schüler sollte diese
  vorgegeben werden. Bei sehr engagierten und älteren Schülerinnen und
  Schülern können auch größere Teile hiervon selbst entwickelt werden,
  dies wird aber im Vorfeld eine intensivere Besprechung erfordern.
  Generell soll der Ball realistisch von den Wänden abprallen können. Am
  einfachsten zerlegt man hierzu die Bewegung des Balls in eine Bewegung
  in Richtung der horizontalen x-Achse und der vertikalen y-Achse. Beim
  Abprallen an der oberen und unteren Wand wird das Vorzeichen in
  y-Richtung umgekehrt, bei Zusammenprall mit einem Schläger wird das
  Vorzeichen in x-Richtung umgekehrt.
\item
  \emph{Zählen der Punkte:} Einen Punkt erhält der Spieler dann, wenn
  der Ball am gegenerischen Schläger vorbei an die hintere Wand anstößt.
  Den einfachsten Punktezähler erhält man, wenn man einen farbigen
  Strich ans Ende des Spielfeldes setzt und den Ball auf einen
  Zusammenstoß mit ebendiesem überprüft. Wird ein solcher Zusammenstoß
  detektiert, so erhält der entsprechende Spieler einen Punkt und der
  Ball wird wieder zurück an die Startposition gesetzt.
\end{enumerate}

\subsubsection{Phase 4: Testen und
vorstellen}\label{phase-4-testen-und-vorstellen}

In der letzten Phase, die auch zweimal durchgeführt werden kann, müssen
alle (bei großen Gruppen: ausgewählte) Schülerinnen und Schüler ihr
Projekt vorstellen. Hier bietet es sich an, wenn eine dritte Person das
Programm bedienen muss und nicht die Programmiererin oder der
Programmierer. Der restliche Kurs schaut bei der Vorstellung zu und gibt
hinterher eine Rückmeldung. Für jedes Projekt kann so analysiert werden,
was gut und was weniger gut funktioniert und noch verbessert werden
muss. Wichtig vor allem bei jüngeren Klassenstufen: Die guten Teile des
Programmes besonders hervorheben!

Es kann anschließend eine weitere Programmierphase angestoßen werden,
bei der jedem die Chance gegeben wird, die vorhandenen Mängel noch zu
beheben.

\subsection{Erweiterungen}\label{erweiterungen}

\begin{itemize}
\item
  \emph{Erschwerte Bedingungen:} Mit jedem Punkt Differenz erhöht sich
  die Schwierigkeit für den führenden Spieler. Dies könnte z. B.
  passieren, indem seine Schlagfläche verkleinert oder die
  Bewegungsgeschwindigkeit des Schlägers verringert wird.
\item
  \emph{\ldots{}}
\end{itemize}
