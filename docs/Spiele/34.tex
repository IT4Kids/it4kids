\section{Katze fängt Maus}\label{katze-fuxe4ngt-maus}

Erforderliche Kompetenzen: Steuerung, Bedingungen, Schleifen, Bewegung

Gewonnene Kompetenzen: Kostüme, Nachrichten

\begin{figure}[ht]
    \centering 
    \includeimage{img/34.png}
    \caption[\Sectionname]{\Sectionname}
\end{figure}

\subsection{Beschreibung}\label{beschreibung}

In diesem Spiel steuert der Spieler eine Katze, die eine davonlaufende
Maus jagt und frisst, falls sie diese fängt.

\subsection{Durchführung}\label{durchfuxfchrung}

\subsubsection{Phase 1: Planung}\label{phase-1-planung}

Zu Beginn soll das Programm an der Vorlage erläutert und durch die
Schüler getestet werden. Dabei kann auf folgende Programmbestandteile
aufmerksam gemacht werden:\\
-Die Steuerung der Katze durch den Spieler\\
-Die sich bewegenden Beine der Katze\\
-Die automatische Flucht der Maus\\
-Das Erscheinen und Verschwinden der Maus

\subsubsection{Phase 2: Vorbereitungen}\label{phase-2-vorbereitungen}

Nun soll die neuen Kompetenzen erläutert und getestet werden. Dazu
können kleine Beispielprogramme oder kleine Programmieraufgaben
verwendet werden.

Als Vorlage für dieses Projekt wird ein Template verwendet, der bereits
den Block ``Lauf weg'' der Maus und das Schrumpfen der Figuren im oberen
Spielfeldbereich enthält. (Lässt sich aus Vorlagen entnehmen)

\subsubsection{Phase 3: Programmierung}\label{phase-3-programmierung}

\emph{Die Bewegung der Katze programmieren:} Zunächst soll die Katze
programmiert werden. Durch Drücken der Pfeiltaste nach oben, soll die
Katze einen Schritt gehen und ihr Kostüm wechseln. Durch Drücken der
Pfeiltasten links und rechts, soll sich die Katze drehen. Am einfachsten
ist dies durch eine ``wiederhole bis''-Schleife zu realisieren. Als
Bedingung wird die Berührung mit der Maus eingesetzt.

\emph{Interaktion zwischen Katze und Maus programmieren:} Beim Starten
des Programmes soll die Katze ihren Hunger auf Mäuse bekunden und
anschließend die Nachricht ``Lauf weg'' an alle versenden, um der Maus
den Beginn der Jagd mitzuteilen.

Die Maus soll nun bei Erhalten der Nachricht sichtbar werden (``zeige
dich''-Block) und den bereits definierten Block ``Lauf weg'' ausführen.

Außerdem soll nach dem Kontakt mit der Maus, der ebenfalls detektiert
werden muss, (z.Bsp. durch die ``wiederhole bis''-Schleife) die
Nachricht ``Ich hab dich gefressen!'' versendet werden.

Die Maus wiederum soll auf diese Nachricht mit reagieren, indem sie sich
versteckt (``versteck dich''-Block) und alle Skripte dieser Figur
stoppt.

\subsubsection{Phase 4: Testen und
vorstellen}\label{phase-4-testen-und-vorstellen}

In der letzten Phase, die auch zweimal durchgeführt werden kann, können
alle (bei großen Gruppen: ausgewählte) Schülerinnen und Schüler ihr
Projekt vorstellen. Hier bietet es sich an, wenn eine dritte Person das
Programm bedienen muss und nicht die Programmiererin oder der
Programmierer selbst. Der restliche Kurs schaut bei der Vorstellung zu
und gibt hinterher eine Rückmeldung. Für jedes Projekt kann so
analysiert werden, was gut und was weniger gut funktioniert und noch
verbessert werden muss. Wichtig vor allem bei jüngeren Klassenstufen:
Die guten Teile des Programmes besonders hervorheben! Es kann
anschließend eine weitere Programmierphase angestoßen werden, bei der
jedem die Chance gegeben wird, die vorhandenen Mängel noch zu beheben.

\subsection{Erweiterungen}\label{erweiterungen}

Das Spiel kann durch eine Zeitmessung erweitert werden, indem durch
Verwendung der Stoppuhr die Zeit gemessen wird, die die Katze braucht,
um die Maus zu fangen.
