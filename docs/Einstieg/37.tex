\section{Zeichnen}\label{zeichnen}

Erforderliche Kompetenzen: Bewegen\\
Gewonne Kompetenzen: Zeichnen, Bedingungen

Ziel des Projekts ``Zeichnen'' ist das tiefere Verständnis des
Koordiatensystems sowie der Drehwinkel. Außerdem werden Möglichkeiten
aufgezeigt, wie Formen auf den Bildschirm gezeichnet werden können.

\begin{figure}[ht]
    \centering 
    \includeimage{img/37.png}
    \caption[\Sectionname]{\Sectionname}
\end{figure}

\subsection{Vorbereitung}\label{vorbereitung}

Das Projekt sollte mit einer Einführung und einem anschaulichen Beispiel
an der Tafel begonnen werden. Dabei wird eine Schülerin oder ein Schüler
zur Tafel gebeten, um ein Quadrat zu zeichnen.

Dabei darf die Person an der Tafel nur den Anweisungen seiner
Mitschülerinnen und Mitschüler folgen. Aufgabe ist es, mittels der
Befehle

\begin{itemize}
\item
  Vorwärts
\item
  Drehe dich (wie weit?)
\end{itemize}

verschiedene Formen zu zeichnen. Für den Beginn eigenen sich Quadrate,
Dreiecke, Sechsecke und später auch ein Kreis.

Ziel des Projekts ist es, dass die Schülerinnen und Schüler mit den
verschiedenen Drehwinkeln vertraut gemacht werden. Mögliche Anweisungen
könnten dann etwa so aussehen:

\begin{itemize}
\item
  Quadrat: ``Vorwärts, Drehen (90 Grad), Vorwärts, Drehen (90 Grad),
  Vorwärts, Drehen (90 Grad), Vorwärts, Drehen (90 Grad)
\item
  Sechseck: Vorwärts, Drehen (60 Grad), Vorwärts, \ldots{}
\item
  Kreis: ein Stück vorwärts, Drehen, ein Stück vorwärts, \ldots{}
\end{itemize}

Diese Anweisungen können natürlich auch hier wie bei ``Informatik mit
Stift und Papier'' durch entsprechende \emph{Wiederhole}-Anweisungen
stark abgekürzt werden.

Welche Formen genau eingeführt werden, bleibt generell dem Kursbetreuuer
überlassen. Der Kreis ist als schwierigste Form anzusehen und kann auch
optional als ``Bonusaufgabe'' beim nun folgenden Programmieren für
besonders schnelle Schüler herangezogen werden.

\subsection{Zum Projekt}\label{zum-projekt}

Nach der Einführung sollten alle Schülerinnen und Schüler die Grundidee
des Projekts erfasst haben. Ziel ist es nun, die zuvor an der Tafel
gezeichneten Formen im Programm umzusetzen. Die Tafel mit dem
entsprechenden Anschrieb kann dabei je nach Stärke der Gruppe
verdeckt/zugeklappt werden oder aber auch offen bleiben und als Vorlage
dienen.

Die Aufgabe am PC ist, mit der Programmierumgebung mindestens drei
verschiedene Formen zu zeichnen.

\subsubsection{\texorpdfstring{Blöcke in der Gruppe
``Zeichnen''}{Blöcke in der Gruppe Zeichnen}}\label{bluxf6cke-in-der-gruppe-zeichnen}

In der Kategorie ``Zeichnen'' sind in der Programmierung alle wichtigen
Befehle zusammengefasst, die den sog. Malstift steuern. Dieser virtuelle
Stift greift im Mittelpunkt der jeweiligen Figur an und kann abgesetzt
und aufgenommen werden. Wenn der Stift abgesetzt ist, zieht die Figur
bei ihren Bewegungen eine Linie hinter sich her. Diese Linie soll hier
genutzt werden, um die Formen zu zeichnen.

\subsubsection{\texorpdfstring{Blöcke in den Gruppen ``Steuerung'' und
``Bewegung''}{Blöcke in den Gruppen Steuerung und Bewegung}}\label{bluxf6cke-in-den-gruppen-steuerung-und-bewegung}

Mit den Blöcken ``gehe (10)er Schritt'' und ``drehe dich um (15) Grad''
werden die Bewegungen realisiert. Zu beginn können mehrere dieser Blöcke
aneinandergereiht werden. Wenn die Schüler schnell zurecht kommen, kann
das Programm gekürzt werden, indem mehrere dieser Blöcke über die
Steuerungsblöcke ``wiederhole (10) mal'' zusammengefasst werden.\\
Die Geschwindigkeit der Bewegung kann über die Variation von
Wiederholungen und Schritten eingestellt werden.\\
Beispielsweise führen folgende Möglichkeiten zu der gleichen Strecke,
bei der ersten Möglichkeit bewegt sich die Figur aber schneller:

\begin{itemize}
\item
  wiederhole 10 Mal: gehe 10er Schritt
\item
  wiederhole 100 Mal: gehe 1er Schritt
\end{itemize}

\subsection{Erweiterungen}\label{erweiterungen}

\begin{itemize}
\tightlist
\item
  \emph{Farben}: eine Schülerin, welche drei oder mehr geometrische
  Figuren programmiert hat, könnte nun ein Hintergrundbild zeichnen, wo
  es drei oder mehr farbige Bereiche gibt. Der Charakter soll dann je
  nachdem auf welcher Farbe er sich geradet befindet, eine bestimmte
  geometrische Figur zeichnen. Hier könnte der Befehl ``Wenn ich
  angeklickt werde'' zum Einsatz kommen. Dies eignet sich zur Einführung
  von Bedingungen.
\end{itemize}
