\section{Fang mich!}\label{fang-mich}

Erforderliche Kompetenzen: Bewegungen, Steuerung, Fühlen, Senden und
Empfangen\\
Gewonnene Kompetenzen: Stoppuhr, Variablen

\begin{figure}[ht]
    \centering 
    \includeimage{img/35.png}
    \caption[\Sectionname]{\Sectionname}
\end{figure}

\subsection{Beschreibung}\label{beschreibung}

Die Katze und der Käfer jagen einander. Dabei tauschen sie die Rollen,
sobald der Jäger sein Ziel gefangen hat, oder 30 Sekunden abgelaufen
sind. Dabei soll der Käfer computergesteuert, aber die Katze
spielergesteuert sein. Es gibt höchstens fünf Punkte zu erreichen.

\subsection{Durchführung}\label{durchfuxfchrung}

\subsubsection{Phase 1: Planung}\label{phase-1-planung}

In der Planungsphase werden mit den Schülern die folgenden Dinge
besprochen:

\begin{itemize}
\item
  Wie programmiert man den Käfer so, dass er die Katze fängt?
\item
  Wie programmiert man den Spielablauf, d.h. dass sich Katze und Käfer
  abwechselnd fangen?
\end{itemize}

\subsubsection{Phase 2: Vorbereitungen}\label{phase-2-vorbereitungen}

Die Schüler laden sich zwei beliebige Figuren aus der Scratchbibliothek
herunter oder zeichnen eigene Figuren. Das Gleiche gilt für den
Hintergrund.

\subsubsection{Phase 3: Programmierung}\label{phase-3-programmierung}

Die benötigten Schritte der Programmierung umfassen:

\begin{itemize}
\item
  Steuerung der Maus per Tastatur
\item
  Automatische Steuerung des Käfers, sodass er die Katze jagt
\item
  Erkennen, dass die Katze den Käfer bzw. der Käfer die Katze gefangen
  hat
\item
  Rollentausch implementieren, d.h. das der bisherige Jäger nun der
  Gejagte ist
\item
  gegebenenfalls: Stoppuhr und Punktezählung
\end{itemize}

\subsubsection{Phase 4: Testen und
vorstellen}\label{phase-4-testen-und-vorstellen}

In der Vorstellungsphase kann ein beliebiger Schüler sein Spiel
vorstellen, falls er es möchte. Dabei kann man explizit auf die
einzelnen Teilprobleme hinweisen, welche der Schüler gelöst hat. Aber
man kann die anderen auch fragen, was er verbessern könnte, wenn er an
diesem Projekt nächste Stunde noch arbeiten will.

\subsection{Detaillierte
Programmbeschreibung}\label{detaillierte-programmbeschreibung}

Die Bewegung ist die fortlaufend wiederholte Ausführung von ``falls
dann''-Blöcken realisiert, die einen Tastendruck detektieren.
Grundsätzlich hat die Figur ``Spieler'', d.i. die Katze, zwei Blöcke für
die zwei Fälle ``when I receive CompiFolgtSpieler'' sowie ``when I
receive SpielerFolgtCompi'', welche sehr ähnlich ablaufen, deshalb nur
die Erklärung für den Ersteren: Die Katze gleitet zur Anfangsposition
(links unten) und solange sie den Käfer noch nicht berührt hat, führt
sie die Bewegungen aus, welche mit der Tastatur an sie weitergeleitet
werden. Dabei wird die Zeit aktualisiert und falls die Zeit abläuft, hat
die Katze gewonnen (weil sie die Weglaufende war) und kriegt deshalb
einen Punkt. Dann wird die Nachricht ``SpielerFolgtCompi'' gesendet,
damit die Figuren die Rollen tauschen und das aktuelle Skript wird
gestoppt, damit die Katze den Code fürs Fangen (anstatt fürs Weglaufen)
ausführen kann. Falls die Zeit aber nicht abläuft und die Schleife
verlassen wird, also die Katze gefangen wurde, wird ein Punkt abgezogen
und die Nachricht ``SpielerFolgtCompi'', d.i. die Abwechslung, wird
gesendet. Wenn allerdings die Katze bereits -5 Punkte hat, erscheint ein
``Verloren!'' und das Spiel wird mit ``stop all'' komplett terminiert.

Die Figur ``Compi'' (der Käfer) macht grundsätzlich das Gleiche mit zwei
Ausnahmen: Der Käfer hat eine Methode ``Fange Spieler!'', welche nur die
eigene Ausrichtung in Richtung Katze setzt und darauf losmarschiert. Die
zweite Ausnahme ist die Methode ``Zufällig weglaufen'': Alle 250
Millisekunden wird hier die Richtung des Käfers um verändert.
