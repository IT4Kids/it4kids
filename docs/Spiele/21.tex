\section{Was soll ich anziehen?}\label{was-soll-ich-anziehen}

Erforderliche Kompetenzen:

\begin{figure}[ht]
    \centering 
    \includeimage{img/21.png}
    \caption[\Sectionname]{\Sectionname}
\end{figure}

\subsection{Beschreibung}\label{beschreibung}

Dies ist ein Spaßprojekt: Auf der Bühne gibt es eine Figur und eine
Menge von Kleidungsstücken. Die Figur soll vom Spieler angezogen werden,
indem dieser auf die verschiedenen Kleidungsstücke klickt.

\subsection{Durchführung}\label{durchfuxfchrung}

\subsubsection{Phase 1: Planung}\label{phase-1-planung}

\begin{itemize}
\item
  Präsentation eines komplett fertigen Projekts
\item
  Überlegungen und Ideensammlung für Hintergrund, Figur, etc.
\item
  Erläuterung des ``Gehe-zu''-Befehls
\item
  Wie müssen sich die Figur einerseits und die Kleidungsstücke
  andererseits verhalten?
\end{itemize}

\subsubsection{Phase 2: Objekte und Hintergrund
zeichnen}\label{phase-2-objekte-und-hintergrund-zeichnen}

Zeichnen von Hintergründen, Figuren etc. Man kann, wie es im Beispiel
gezeigt ist, auch den Hintergrund weiß lassen, aber besonders spaßig ist
es nicht.

\subsubsection{Phase 3: Programmieren}\label{phase-3-programmieren}

\begin{enumerate}
\item
  Das Erscheinen von Figur und Kleidungsstücken
\item
  Die korrekte Bewegung des jeweiligen Kleidungsstücks zur Figur auf
  Klick
\end{enumerate}

\subsubsection{Phase 4: Testen und
vorstellen}\label{phase-4-testen-und-vorstellen}

Wie immer.
