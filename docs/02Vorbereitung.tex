\chapter {Vorbereitung}

\section{Zu den Projekten}
Die Projekte gliedern sich in verschiedene Kategorien. Ziel ist es nicht, eine starre Unterrichtsstruktur vorzugeben. Aus bisheriger Erfahrung lässt sich sagen, dass die Auswahl der Materialien und Themen für den Kurs sehr entscheidend von der Kurszusammensetzung abhängt und dass eine korrekte Auswahl unabdingbar für das Gelingen des Kurses sein kann.

Aus diesem Grunde verfolgen wir in diesem Handbuch ein modulares Konzept: Die Projekte sind nicht allgemein nach Niveau (Anfänger, Fortgeschritten etc.) gegliedert, sondern in verschiedene Kompetenzbereiche unterteilt. Es gibt verschiedene Möglichkeiten, alle Bereiche abzudecken.

Trotz dieser bewusst offen gehaltenen Grundstruktur sollen an dieser Stelle einige beispielhafte Abläufe genannt werden, um die Kursplanung zu vereinfachen und einen schnellen Start zu ermöglichen.

\subsection{Kursvorschlag: Grundschule}

\begin{itemize}
	\item Einstieg: Labyrinth im Klassenraum %TODO link
    \item Erstes Projekt: Maus zum Käse
    \item Geschichten
    \item Klänge
    \item Aquarium
\end{itemize}

\subsection{Kursvorschlag: Weiterführende Schule}
\begin{itemize}
	\item Einstieg: Labyrinth im Klassenraum %TODO link
    \item Erstes Projekt: Labyrinth
    \item Einführungsprojekte
    \begin{itemize}
        \item Zeichnen und Farben
        \item Klänge
        \item Botschaften (zB Tanzstunde oder Bananen)
    \end{itemize}
    \item Wettrennen
    \item Pong
\end{itemize}


%TODO Schreiben

\section{Notwendige Tools}
Für die Durchführung der hier vorgestellten Projekte im Kurs kommen verschiedene Werkzeuge infrage, die im Folgenden vorgestellt werden sollen.

\subsection{Scratch Editor}
Scratch ist eine am MIT entwickelte, quelloffene Programmierumgebung. Scratch ist sowohl als Online- als auch als Offline-Version frei erhältlich.

es sowohl online als auch offline. Scratch ist die Standardumgebung, auf die für die Projekte in diesem Konzept zurückgegriffen wurde. Auf Scratch kann online unter \href{https://scratch.mit.edu}{scratch.mit.edu} zugegriffen werden.

\begin{figure}[ht]
	\centering
		\includeimage{img/Scratch}
	\caption[Scratch Online Editor]{Scratch Online Editor}
	\label{fig:scratch}
\end{figure}

Alle hier genannten Vorlagen mitsamt Beschreibungstext und Musterlösung sind online unter \href{http://www.it-for-kids.org/projects}{it-for-kids.org/projects} abrufbar. Zudem liegen die aktuellen Projektstände auf der mitgelieferten CD bereit.

Das Laden von Projekten in Scratch erfolgt durch Druck auf ''Entwickeln`` auf der Startseite. Wir empfehlen, im Kursbetrieb diese Seite als Lesezeichen oder Verknüpfung auf den Schülerrechnern zu hinterlegen.


\subsection{IT4Kids Online Editor}
Als Alternative zu Scratch sind wir bemüht eine eigene, auf die konkreten Bedürfnisse des Kursbetriebes zugeschnittene Entwicklungsumgebung anzubieten. Die Umgebung befindet sich derzeit in der aktiven Entwicklung und ist eine veränderte Version der ebenfalls quelloffenen Software \textit{Snap!}. Auf den IT4Kids Online Editor kann unter \href{http://code.it-for-kids.org}{code.it-for-kids.org} zugegriffen werden.

\subsection{IT4Kids Offline Editor}
Ebenfalls in der aktiven Entwicklung befindet sich der IT4Kids Offline Editor. Der aktuelle Entwicklungsstand kann auf unserer Homepage eingesehen werden.

\section{Vor Kursbeginn}

Vor Beginn eines IT4Kids Kurses empfehlen sich einige Vorkehrungen, die in diesem Abschnitt besprochen werden.

\subsection{Verfügbarkeit eines Beamers}
Bei vielen Projekten bietet es sich an, die fertigen Programme von den Schülerinnen und Schülern präsentieren zu lassen. Außerdem eignet sich ein Beamer sehr gut, um zu Beginn des Kurses das Stundenziel sowie das fertige Projekt zu zeigen. Sollte kein Beamer zur Verfügung stehen können Projekte an einem gut einsehbaren Rechner gezeigt werden. Manche Konzepte lassen sich auch gut an einer Tafel oder einem OHP demonstrieren.

\subsection{Infrastruktur zum Abspeichern von Projekten}
Zum Abspeichern von Projekten eignen sich grundsätzlich zwei Ansätze: Scratch unterstützt das manuelle Hoch- und Herunterladen von Projekten. Sofern an der Schule ein Netzwerkordner vorhanden ist, kann dieser für die Verteilung von Vorlagen und auch zum Abspeichern von Projekten verwendet werden.

Der zweite Ansatz ist eine Anmeldung auf der Scratch Homepage. Hier kann sich jede Schülerin und jeder Schüler einen eigenen Account zulegen. Einmal angemeldet, wird das jeweils aktuelle Projekt jeweils automatisch in der Cloud gespeichert.

In jedem Fall sollte im Vorfeld mit der Schule die Verfügbarkeit eines Netzwerkordners abgestimmt werden.

Auch die Verwendung von Schüler-USB-Sticks ist denkbar, sodass jeder Schüler seine Fortschritte für zuhause speichern kann. Es empfiehlt sich jedoch, die Projekte auch noch an einem anderen Ort zu speichern, da die SChüler ihre Sticks u.U. zuhause vergessen.

\subsection {Vor Stundenbeginn}
Als Kursbetreuer sollte vor dem Kurs das Projekt ausgewählt und verstanden werden. Dazu reicht es in der Regel sich das Lehrkonzept durchzulesen und die Programmierung in der Musterlösung nachzuvollziehen. Bei komplexeren Projekten kann es hilfreich sein sich bestimmte Programmteile zu notieren oder auf ein Cheat-sheet zurückzugreifen.
Erfahrungsgemäß ist es als Kursbetreuer ratsam sich deutlich vor Stundenbeginn im Kursraum einzufinden. So können die bisweilen etwas langsameren Schulrechner in aller Ruhe hochgefahren werden und auch Projekte und Scratchumgebung geöffnet werden. Dies kann andernfalls eine nicht unbeachtliche Zeit in Anspruch nehmen, was unter den Schülern zu Unruhe oder Unmut führt.
