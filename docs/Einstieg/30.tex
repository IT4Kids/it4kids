\section{Labyrinth}\label{labyrinth}

\textbf{Erforderliche Kompetenzen:}\\
keine

\textbf{Gewonnene Kompetenzen:}\\
Scratch-Umgebung kennen lernen

Ereignisse \{Startbedingung (Wenn angeklickt)\}

Bewegung \{Koordinaten (gehe, gehe in zu), Winkel (drehe dich)\}

Steuerung \{Wiederholungen (wiederhole)\}

\begin{figure}[ht]
    \centering 
    \includeimage{img/30.png}
    \caption[\Sectionname]{\Sectionname}
\end{figure}

\subsection{Beschreibung}\label{beschreibung}

Mit dem Start des Programms soll die Maus ohne weitere Eingabe das
Labyrinth bis zum Ziel durchlaufen.\\
Bei der Schlüsselvariante muss die Maus vorher auf den Schlüssel
navigieren, damit sich das Tor öffnet.

\subsection{Durchführung}\label{durchfuxfchrung}

\subsubsection{Phase 1: Planung}\label{phase-1-planung}

In der Planungsphase werden mit den Schülern die folgenden Punkte
besprochen:

\begin{itemize}
\tightlist
\item
  Wie und wodurch wird ein Programmteil gestartet? Welchen Einfluss hat
  dies auf andere Programmteile?
\item
  Wie Programmiere ich eine spezielle Figur und nicht versehentlich eine
  andere?
\item
  Was sind Koordinaten und Winkel? Wie kann ich diese ablesen? (Dieser
  Punkt ist optional, gerade in der Grundschule kann das Verständnis von
  Gradzahlen/Winkeln und Koordinaten auf einen späteren Zeitpunkt mit
  einem spezifischen Projekt verschoben werden.)
\end{itemize}

\subsubsection{Phase 2: Vorbereitungen}\label{phase-2-vorbereitungen}

Hierbei ist eine Einarbeitung vorgesehen: Wo bearbeitet man die Figuren,
Info der Figuren, Koordinaten und Winkel ablesen, Wo sind die Skripte
gehe, drehe dich, \ldots{} zu finden.

\subsubsection{Phase 3: Programmierung}\label{phase-3-programmierung}

Die verwendeten Befehle sind:

\begin{itemize}
\tightlist
\item
  wiederhole \ldots{} mal
\item
  gehe \ldots{} er-Schritt
\item
  drehe dich um \ldots{} Grad
\item
  Diese Schritte müssen so oft wiederholt werden, bis die Maus das Ziel
  erreicht hat.
\end{itemize}

\subsubsection{Phase 4: Testen und
vorstellen}\label{phase-4-testen-und-vorstellen}

In der Vorstellungsphase können die Schüler ihre Lösung vorstellen,
wodurch die unterschiedlichen Arten der Lösung erläutert werden können.
Dabei sollte jedoch der spielerische Spaß im Vordergrund stehen und
nicht abschreckend wirken.

\subsection{Erweiterungen}\label{erweiterungen}

Als Erweiterung zum Labyrinth steht noch Labyrinth mit Schlüssel zur
Verfügung, in diesem Können sie sich weiter mit der Problematik
auseinander setzen und eine andere Variation der
Programmiermöglichkeiten aneignen.
