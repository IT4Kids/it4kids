\section{Vertiefungen}\label{vertiefungen}

Wie bereits am Anfang dieses Kapitels gesagt lassen sich mit animierten
Geschichten ganz gezielt bestimmte Fähigkeiten erlernen.

Es können hierzu eigene Geschichten extra konzipiert werden, die Wert
auf eine bestimmte Aktion legen. Dadurch können die Schüler kreativ die
Geschichte gestalten,und lernen dennoch neue Fähigkeiten. Diese Form von
Geschichte eignet sich vor allem, wenn bereits ein gewisses
grundlegende Verständnis besteht, aber noch nicht alle Funktionen von Scratch
erlernt sind. Die Projekte ``Maus zum Käse'' und ``Tanzstunde'' fallen
mehr oder weniger auch unter diese Form von Animation.

Einige Beispiele für Anwendungen von Geschichten im Bezug auf besondere
Fähigkeiten:

\begin{itemize}
\item
  Um das Verständnis von Kostümen zu erhöhen: Eine Geschichte über
  Figuren, die sich verwandeln (egal ob willentlich oder nicht).
  Hierdurch muss verstanden werden, was der Unterschied zwischen dem
  Hinzufügen neuer Figuren und dem Hinzufügen von Kostümen zu einer
  bestehenden ist.
\item
  Für das Erlernen von Nachrichten ist ein Szenario in dem miteinander
  Gesprochen wird oder Befehle ausgegeben werden sinnvoll, da der
  Gedanke einfach auf eine Nachricht zu übertragen ist und das Konzept
  schnell klar wird.
\end{itemize}
