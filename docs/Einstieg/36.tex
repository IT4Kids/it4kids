\section{Tanzstunde}\label{tanzstunde}

Erforderliche Kompetenzen: Scratch-Umgebung kennen\\
Gewonnene Kompetenzen: Startbedingungen, Nachrichten, Wiederholungen,
Kostüme, Klänge/Instrumente

\begin{figure}[ht]
    \centering 
    \includeimage{img/36.png}
    \caption[\Sectionname]{\Sectionname}
\end{figure}

\subsection{Beschreibung}\label{beschreibung}

Wird die oberste Figur, der Tanzlehrer, angeklickt, sendet er eine
Nachricht an alle Tänzer, das sind die anderen drei Figuren unter ihm.
Beim empfangen dieser Nachricht beginnen die Tänzer, zu tanzen, und
Musik zu machen. Die Tänzer tanzen, indem sie ihre Kostüme wechseln.

\subsection{Durchführung}\label{durchfuxfchrung}

\subsubsection{Phase 1: Planung}\label{phase-1-planung}

In der Planungsphase werden mit den Schülern die folgenden Punkte
besprochen:

\begin{itemize}
\item
  Wie wird eine Nachricht (``Message'') versandt und empfangen
\item
  Wie werden einer Figur mehrere Kostüme zugewiesen und diese im
  Programm geändert
\item
  Welche Möglichkeiten gibt es, Klänge zu erzeugen
\end{itemize}

\subsubsection{Phase 2: Vorbereitung}\label{phase-2-vorbereitung}

In der Vorbereitungsphase erstellen die Schüler die nötigen Figuren.
Diese können auch noch beliebig bearbeitet und kreativ veränder werden,
in z.Bsp.: weitere Kostüme hinzugefügt oder Figuren selber gemalt
werden.

Eine oder mehrere Bühnenbilder sind auch denkbar, welche dann zusätzlich
mit der Musik variieren können.

\subsubsection{Phase 3: Programmierung}\label{phase-3-programmierung}

\begin{itemize}
\item
  sende / empfange Message
\item
  nächstes Kostüm
\item
  warte \ldots{} Sek.
\item
  diese letzten beiden Schritte werden fortlaufend wiederholt
\item
  zwischen den Wiederholungen kann noch mit Klang gearbeitet werden
\end{itemize}

\subsubsection{Phase 4: Testen und
vorstellen}\label{phase-4-testen-und-vorstellen}

In der Vorstellungsphase können die Schüler ihre eigenständig
Individuellen Kreationen vorstellen. Dabei haben die Schüler die
Möglichkeit sich gegenseitig zu inspirieren.

\subsection{Erweiterungen}\label{erweiterungen}

Als Erweiterung dazu können die Schüler noch unter Daten eine Variable
hinzufügen, die die Anzahl der Klicks vom Tänzer zählt.
