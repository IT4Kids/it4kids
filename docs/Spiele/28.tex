\section{Wettrennen}\label{wettrennen}

Erforderliche Kompetenzen: Bewegungen, Fühlen, Bedinungen

Erworbene Kompetenzen: Spieleprogrammierung, Variablen

\begin{figure}[ht]
    \centering 
    \includeimage{img/28.png}
    \caption[\Sectionname]{\Sectionname}
\end{figure}

\subsection{Beschreibung}\label{beschreibung}

Dieses Projekt eignet sich sehr gut zum Einstieg in die
Spieleentwicklung, denn es handelt sich um ein einfaches dass zwei
Spieler gegeneinander auf einer Tastatur spielen können. Bei dem Projekt
sollen zwei Spielfiguren vom linken zum rechten Bildrand ''um die Wette
laufen''. Die Steuerung des Spiels gestaltet sich dabei einfach: Jeder
Spieler muss möglichst schnell abwechselnd zwei ihm zugeordnete Tasten
drücken. Bei jedem Tastendruck bewegt sich die Figur einen Schritt in
Richtung Ziel am rechten Bildrand.

\subsection{Durchführung}\label{durchfuxfchrung}

\subsubsection{Phase 1: Planung}\label{phase-1-planung}

Während der Planungsphase sollten mit den Schülerinnen und Schülern die
folgenden Schritte vollzogen werden:

\begin{itemize}
\item
  Erläuterung des Projektes
\item
  Überlegungen und Ideensammlung für Hintergrund, Figuren, etc.
\item
  Überlegungen zur Programmierung und der nötigen Aufgaben:

  \begin{itemize}
  \item
    Bewegung der Spieler/Steuerung
  \item
    Erkennen des Zieldurchlaufs
  \end{itemize}
\item
  Sammeln aller Ideen und Erstellung der TODO Liste
\end{itemize}

Anschließend kann mit der Umsetzung des Projekts begonnen werden.

\subsubsection{Phase 2: Vorbereitungen}\label{phase-2-vorbereitungen}

Zunächst sollten die Figuren und der Hintergrund gezeichnet werden. Für
die Figuren bieten sich einfache Striche an, die auf und ab bewegt
werden können, die Gestaltung des Hintergrunds ist den Schülerinnen und
Schülern überlassen.

\subsubsection{Phase 3: Programmierung}\label{phase-3-programmierung}

Im Anschluss an die vorbereitenden Maßnahmen kann mit der Programmierung
des Spiels begonnen werden. Als Vorschlag sei die folgende Reihenfolge
genannt, in der Schritt für Schritt leicht validierbare Teile des Spiels
entstehen:

\begin{enumerate}
\item
  \emph{Steuerung der Figuren:} Die Figuren sollen jeweils auf zwei
  Tasten mit einem 10er-Schritt reagieren. Allerdings sollen die Tasten
  abwechselnd betätigt werden müssen.
\item
  \emph{Erkennung des Ziels:} Ob eine Figur das Ziel erreicht hat, kann
  über die Bedingung realisiert werden, ob die Farben des Zielbanners
  berührt werden.
\item
  \emph{Ende des Spiels:} Sobald eine Figur das Ziel erreicht hat, soll
  das Spiel beendet werden. Dies ist einfach mit dem Block ``Stoppe
  alles'' zu realisieren.
\end{enumerate}

\subsubsection{Phase 4: Testen und
vorstellen}\label{phase-4-testen-und-vorstellen}

In der letzten Phase, die auch zweimal durchgeführt werden kann, müssen
alle (bei großen Gruppen: ausgewählte) Schülerinnen und Schüler ihr
Projekt vorstellen. Hier bietet es sich an, wenn eine dritte Person das
Programm bedienen muss und nicht die Programmiererin oder der
Programmierer. Der restliche Kurs schaut bei der Vorstellung zu und gibt
hinterher eine Rückmeldung. Für jedes Projekt kann so analysiert werden,
was gut und was weniger gut funktioniert und noch verbessert werden
muss. Wichtig vor allem bei jüngeren Klassenstufen: Die guten Teile des
Programmes besonders hervorheben!

Es kann anschließend eine weitere Programmierphase angestoßen werden,
bei der jedem die Chance gegeben wird, die vorhandenen Mängel noch zu
beheben.

\subsection{Erweiterungen}\label{erweiterungen}

Das Spiel kann um einen Schiedsrichter erweitert werden, der einen
Countdown herunterzählt und anschließend eine Nachricht verschickt, die
die Bewegung der Figuren ermöglicht.
